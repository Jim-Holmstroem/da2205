\documentclass[a4paper,twoside=false,abstract=false,numbers=noenddot,
titlepage=false,headings=small,parskip=half,version=last]{scrartcl}
\usepackage{header}
\usepackage{probability}
\begin{document}
\generateheader{Seminar preparation assignment 2}

\begin{exercise}{1} 
    Are Jacobson and Tellefsen testing any hypothesis? Is so, which one?
\end{exercise}
\begin{solution}
    The original hypothesis that they intend to test is: 
    When a person concentrates vividly on an
    physical object in his surrounding he creates a so called 'psi track' which
    can be detected by dowsing. 
    But since they haven't compared it
    statistically to the $H_0$ I wouldn't consider it as testing. They also add
    a few ad-hoc hypothesis but those aren't tested just added to basically
    save the theory.
\end{solution}

\begin{exercise}{2} 
    How were the experiments designed? Describe the experimental setup briefly.
    In your description, include answers to the following sub-questions:
    \begin{itemize}
        \item How were the test persons selected?
        \item What objects were hidden?
        \item How were the hiding places selected?
        \item Were Jacobson's and Tellefsen's experiments performed under
        double blind conditions, and if so, what does that mean?
    \end{itemize}
\end{exercise}
\begin{solution}

    Mostly experienced dowsers where chosen as dowsers.

    Mostly small  personal belongings where hidden since it turned out 
    that objects with a
    strong positive connection to the person had strong psi-track where
    chosen, but this was not included in the original hypothesis. 
    
    The hiding places where selected such that the objects couldn't bee seen
    from the sender as well as not be seen until within one meter.

    The objects where not put out by going in a straight path to avoid the
    possibility to track a trail of the placer. 

    Additionally they mention the possibility for fraud amongst the test
    persons, for example the sender and dowser coming up with a hiding place in
    advance. Personally I think missing belongs and finding a thief isn't to
    hard, for example point at the closest large city will suffice being a good
    guess.

    I think the condition that no one in the experiment setup knows the right
    answer is fulfilled and can thus be considered to be double blind. Some of
    the experiments where just testing and those where a bit vague on
    this aspect.

\end{solution}
\begin{exercise}{3} 
    Were the results statistically evaluated?
\end{exercise}
\begin{solution}
    No, they only collected data, since doing a statistical evaluation
    would require to statistically reject the null hypothesis by
    checking if the $p$-value exceed some threshold say $5\%$ then would could have
    rejected the null hypothesis that the "psi track" doesn't exist.
\end{solution}

%-----------------------
\end{document}
