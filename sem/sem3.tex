\documentclass[a4paper,twoside=false,abstract=false,numbers=noenddot,
titlepage=false,headings=small,parskip=half,version=last]{scrartcl}
\usepackage{header}
\usepackage{probability}
\begin{document}
\generateheader{Seminar preparation assignment 3}

\begin{exercise}{1a}
    What was the experiment meant to find out?
\end{exercise}
\begin{solution}
    Check whether light is propagating trough a stationary medium or not.
\end{solution}
\begin{exercise}{1b}
    Describe the experimental setup. In what way could this setup provide a
    test of the investigated hypothesis?
\end{exercise}
\begin{solution}
    They basically measured the difference in traveling time between 
    traveling directions. This was achieved by letting the light from one
    source be split in a half-transparent mirror to go different directions and
    then measure the inference on the way back to see the small possible time
    difference between the two directions. This must of course be calibrated
    properly.
\end{solution}
\begin{exercise}{1c}
    What measures were taken to reduce uncertainty and errors in the
    experiment?
\end{exercise}
\begin{solution}
    Repeating the measurements in different settings like at different times of
    the day as well as different times of the year to avoid having a
    experimental result solely based on the coincident of alignment of the Earth.
\end{solution}
\begin{exercise}{1d} 
    What was the outcome of the experiment?
\end{exercise}
\begin{solution}
    No time difference between the two light beams traveling in the different
    directions.
\end{solution}
\begin{exercise}{1e}
    What could be inferred about the hypothesis from the results of the
    experiment?
\end{exercise}
\begin{solution}
    The light speed was invariant of rotation and thereby invariant of the
    velocity of the reference frame. This results in that light doesn't travel
    in a stationery medium.
\end{solution}
\begin{exercise}{1f} 
    How where the results interpreted?
\end{exercise}
\begin{solution}
    Either there is no such thing as ether or the experiment failed to catch
    the effect by having a null result from some effect canceling out the
    difference in time in this particular setup.
\end{solution}
%-----------------------
\end{document}
