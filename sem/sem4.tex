\documentclass[a4paper,twoside=false,abstract=false,numbers=noenddot,
titlepage=false,headings=small,parskip=half,version=last]{scrartcl}
\usepackage{header}
\usepackage{probability}
\begin{document}
\generateheader{Seminar preparation assignment 4}

\begin{exercise}{1}
    What is the difference between fabrication of data falsification of data?
\end{exercise}
\begin{solution} Both are considered as misconduct \\
    \begin{description}
       \item[Fabrication] Making up results and recording or reporting them. \\
       \item[Falsification] Changing or omitting results or manipulating
       equipment or process. \\
    \end{description}
\end{solution}
\begin{exercise}{2}
    Give at least 3 reasons (besides avoiding plagiarism) why you should cite
    other's work in you research?
\end{exercise}
\begin{solution}
    Directly from the paper On being a scientist "They acknowledge the work of
    other scientist, direct the reader toward additional sources of
    information, acknowledge conflicts with other results, and provide support
    for the views expressed in the paper."
\end{solution}
\begin{exercise}{3}
    Are there any ethical problems with pseudoscience (such as homeopathy or
    intelligent design)? And if so, which are they? (In your answer, repeat
    also the criteria for something to be considered to be pseudoscience).
\end{exercise}
\begin{solution}
    Pseudoscience is belief or practice which is presented as scientific but
    are missing a valid scientific method, lacks evidence or cannot be reliably
    tested. 
\end{solution}
%-----------------------
\end{document}
