\documentclass{article}
\usepackage{graphicx}
\usepackage{amsmath}
\usepackage{amssymb}
\usepackage{hyperref}
\usepackage{header}
\begin{document}

\generateheader{Final Assignment}

\section{Summary} %state what book and what part I focused on

\section{Do I agree with the conclusions?}

\section{Am I excited?}

    Artificial Scientist: 
    * Use algorithms with active learning on real life experiments for example the
    microfluid one.
    * Use an artificial scientist to improve the artificial scientist, OMFG

    * Hardware is a bit  behind, need neurla processors or processors which do
    errors but the algorithms are errorcorrecting initself, which is closer to
    neuralprocessing in the neuronnets.
    * also need other ways of thinking dealing with massive-parallel, refer to
    bigdata-eventguy which had had the new parallel-databse idea. Most
    machinelearning algorithms is non-sequencial(is it called this?) but separating
    the problem in a map-reduce fasion isn't always trivial.
    * An important piece of the puzzle is to get machinelearning to be scaleable in
    deeplearning(is this correct usage of deeplearning?) so that we don't have to
    generate features byhand but instead have raw input with minimal preprocessing
    (perhaps only log-scaling some raw input you know have a exponential behavour.

    * Or redesigning the hardware it is running on to be optimal to .. (recursive
    call)

\end{document}
