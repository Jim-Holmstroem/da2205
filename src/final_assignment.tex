\documentclass{article}
\usepackage{graphicx}
\usepackage{amsmath}
\usepackage{amssymb}
\usepackage{hyperref}
\usepackage{header}
\usepackage{natbib}
\begin{document}

\generateheader{Final Assignment\\ \large ``Artificial Scientists and the
needed paradigm shift''}

%"To pass the course you must complete one of the following assignments. For the
%essay submission (~2300 words) remember you will have to support your claims
%with relevant references etc. Also these essays will be made available to other
%members of the course. Remember the intended audience for your reports is the
%other master students in the class."
%"Read the document Towards Science 2020 and write and summarize the findings and
%conclusions in relation to one of the fields they have focused on. Do you agree
%with their conclusions? Are you excited by them? .... "
%
% >>A new bread of scientists
% >The realization of an artificial scientist will most certainly not be possible
% without a paradigm shift in hardware
% 
% * Let it deal with abstract computer science or algebra problems (have
% already been done somewhat) 
% * Connect IO to the real world
% * Connect the machine to itself -> singularity


\section{Summary}

\section{The dawn of new hardware}
    The current processing capabilities for computers has lately degraded to a
    sub-exponential growth which is starting to have consequences such as 
    restricting the use of some algorithms as well as setting limits 
    to the processable size of datasets.
    The sub-exponential growth in computing has been a historical indicator
    that a new paradigm is on it's way which should be the case this time as
    well or at least a indicator on need.%<ref the_five_paradigm_changes_in_computing>
    
    The issue with most current hardware systems is their heavy and naive use of the von
    Neumann architecture. Basically the throughput for the types of processor
    has started to reach, for physical reasons, the end of the line 
    performance wise.

    The duck-tape solution<ref:amd> to this has been keeping the von Neumann
    architecture and just throw in more cores and more machines in parallel.
    This might work well in some big-data applications where the problems is easily
    map-reduceable<ref?> but perform really badly in many other problems. Also
    this scaling becomes very space and power exhaustive since you have the need for
    cooling of all the cores. <ref
    http://www.gotw.ca/publications/concurrency-ddj.htm>

    The other type of solution is to drop the overused von Neumann architecture
    and go for something that is much more throughput oriented. In the 2020
    paper\cite{ms2020} they foresight
    \footnote{1 and 2 years before the official 
    announcement of CUDA resp. openCL.} 
    the use of graphical processing units
    (GPU) for general computing that is problems with non graphical nature like
    solving differential equations or FFT. High end GPU's nowadays have around
    500? stream? processors in parallel which can perform one type of operation
    at a time on 500-element-chunk of the data in parallel, this together with that they
    only read from a read-only memory, which removes the problem of write?-locks , and that all the buses are separated for
    maximum throughput they can in many cases  scale almost linearly 
    on the number of cores
    which ordinary CPU's almost never manages.

    Both scaling up with ordinary CPU's into multi or manycore as well as
    general processing on GPU's requires new libraries

    In the manycore case one solution instead of the classical object oriented
    programming use a pure functional programming as
    for example the language Haskell.%ref companies dealing with large data
    %that has choosed functional programming over object oriented.
    since the pure functional languages 
    have inherited properties like lacking states which is the bigest bandit in
    the write?-lock %ref write lock
    bottleneck as well lacking execution order which makes it trivial to make a
    program run in parallel.

    Neural Processing Unit (NPU)

    Using neural processing units also requires a new programming paradigm to
    operate, not much information on the SyNAPSE NPU nor the development tools
    have been released but I can speculate that it would need some variant of
    logical- or perhaps functional programming.

\section{A new breed of scientists}
    The need of new hardware becomes really apperant both in
    theory and practice when trying to create a general intelligence.
    A few really recent and in many ways successful tries has been made
    like for example: Google's unsupervised catdetector %REF catdetector
    %IBM simulation Synapse (distinguish this with the NPU Synapse)
    These tries has been run on large clusters blabla but would gain a
    significant effeciency boost being run on more natural habitat namely the
    NPU. 

\section{Summary} %state what book and what part I focused on

    This essay will be focusing on the current software and hardware limitations
    towards the development of new machine learning techniques a specially the
    artificial scientist. Much of the claims and discussions will be based on
    the article ``Towards 2020 Science'' from Microsoft Research.\cite{ms2020}

%    It seems that we are currently standing even closer to the brink of a new 
%    computer revolution. New machine learning techniques are starting to be 
%    developed and set off into production. This among other things results in
%    the need of getting massive amounts of computing power.
%    This hinder is nowadays much more focused on how to 
%    distribute the computations
%    since the performance of a single computing unit is starting to 
%    hit the so called
%    heat wall.\cite{ms2020}

%    SUMMARY ON ARTIFICIAL SCIENTIST 

GPU Which is compared to a von neumann processor much more focused on throughput.

\section{The dawn of new hardware}

    Hardware hasn't really been catching up lately with the dataset and algorithms
    being used in modern machine learning, basically the more the merrier when it comes
    to processing power in machine learning. This is mostly due to that the
    increase in clock speed and memory bandwidth has stagnated for physical
    reasons.
    
    In my own experience I see a need for most programs being written at
    a higher abstraction level, for example most forloop constructions can be
    replaced higher order operations like map/reduce.<ref> 

    There is only two solutions to this basic problem to overcome the
    bottleneck and that is to go around it; either by using more cores working in
    parallel or changing the processor architecture.

    Running code on multiple CPU's creates new types of .. 
    cachelock
    changing the algorithms (this is actually an issue with all new types of
    hardware)

    A new hardware that has gained much popularity for general processing is
    GPU's. <what makes them better in many cases?> 
    Libraries dealing with GPGPU has been released not long after the
    article\cite{ms2020} had been released just as they predicted.

    One must not forget the importance of actually having good API's for
    things...

evolutionary (ie selflearning) processes has exponential growth.. <REF>

continuieng the moors law we need to have a new paradigm
each time in history when a paradigm is starting to hit a brick wall and
moores law slowed down there would popup a new one

since the dawn of computation we have basically had exponential speed through
five different paradigms


Recent development in these areas has created a need for something more the
just a massive number cruncher. We would like something that basically works
more like a brain. With basic tasks as recognizing objects or learn from
experience.
\cite{synapse}

"neurosynaptic computing chips" 
"capable of "rewriting"its connections as it
encounters new information."
SyNAPSE processor

"the project will also increase our understanding of neurology -- specially how
cognition works in people."

This new type of computer could possible be perfect for sense,perceive,interact
and recognize which is basically the hardest main ingredient for artificial
scientists to do intelligently.

"Making all this possible is a paradigm shift in the way that we think about
computing and computing architecture."

The basics is that we need to decentralize computing, distribute the components
to be better for parallelization.

Even if it's a dead-end it will most defently set a spark to develop new
hardware paradigms.

>>>> write about the paradigms shifts in computing
http://www.tomshardware.com/news/moores-law-multicore-serial-parallel-programming,10324.html

"As processors and memory grow ever faster, it's becoming more difficult for
the bus to handle the throughput. Our brain don't have that problem."
cognitive computing chip
we basically need to "bypass von Neumann architecture"

%=============================================================
    %http://www.youtube.com/watch?v=BnTUOEwOKYA
    "So I think there is a great architectural mismatch because we built
    computers as we built today for one purpose they worked really well for the
    purpose and then we used the same architecture for doing everything else we
    can do with computing. I think we have today the luxury of going back and
    rethink how we actually want to build computers and computer architectures,
    and this is what cognitive computing is all about." - Dr. Horst Simon

    "the purpose of these neumorphic chips is to build systems that do things 
    efficiently that computers do poorly so current computers are great at
    adding numbers they can add billion of numbers per second, they are just
    fantastically fast but they do really poorly at recognizing peoples faces
    and recognizing objects and other types of things that are brains do really
    well at. ... the hope is to do recognition tasks in a automatic way." - Dr.
    John Arthur

    Basically we want the computers to do people-stuff

    Compare this choosing of architecture with the early stages of flying
    machines, flapping vs. rigid wings
    
    There is only two solutions to overcome this performance bottleneck from the
    core either you put many of them in parallel and try to split the
    computations among them or you change the processor architecture to better
    fit the needs.

%============================================================

    Since the dawn of computers the processors has been of von nuemann type


    Super scalability
    * Hardware isn't really catching up with the datasets, clock speed and
    memory bandwidth has
    stagnated and the only way to increase the computing power is to increase
    the number of cores or change the underlying architecture (as in GPU) but 
    this introduces a lot of new problems one has to re-implement all
    algorithms for the new setup efficiently which is non trivial in many
    cases. Mention cache-locks and

    * Hadoop (with Mahong ofc) is the most widely used framework for
    distributed processing on immense datasets across a cluster. 

    * Non-perfect hardware: many calculations need not to be perfect to be able
    to operate and some research into this has been done: //REF

\section{A new breed of scientists}
    Artificial Scientist: 
    * Use algorithms with active learning on real life experiments for example the
    microfluid one.
    * Use an artificial scientist to improve the artificial scientist, OMFG

    * Hardware is a bit  behind, need neurla processors or processors which do
    errors but the algorithms are errorcorrecting initself, which is closer to
    neuralprocessing in the neuronnets.
    * also need other ways of thinking dealing with massive-parallel, refer to
    bigdata-eventguy which had had the new parallel-databse idea. Most
    machinelearning algorithms is non-sequencial(is it called this?) but separating
    the problem in a map-reduce fasion isn't always trivial.
    * An important piece of the puzzle is to get machinelearning to be scaleable in
    deeplearning(is this correct usage of deeplearning?) so that we don't have to
    generate features byhand but instead have raw input with minimal preprocessing
    (perhaps only log-scaling some raw input you know have a exponential behavour.

    * Or redesigning the hardware it is running on to be optimal to .. (recursive
    call)


"
Gaetano Borriello, a computer scientist at the University of Washington in
Seattle, argues that such widely distributed computing power will trigger a
paradigm shift as great as that brought about by the development of
experimental science itself. "We will be getting real-time data from the
physical world for the first time on a large scale."
" - http://www.nature.com.focus.lib.kth.se/nature/journal/v440/n7083/full/440402a.html



% outline it in this order: (also check which statements/thoughts are mine vs.
% from the paper) remember to always have backup references.
%   
%   Hardware Limitations
%       -Introduction: Hardware isn't really catching up with what we want to
%       compute
%       -Stagnation in clock speed and memory bandwidth and such for one
%       processor unit
%       -Multicore CPU's
%       -Distributed data/compute systems, both colaborative clusters as well as
%       -supercomputers.(manycores that is computers running on 100core+).
%           Compute
%               Folding@home ("social" computing)
%           data
%               ?
%       -(but the efficient can be as low as 1% <ref>)
%       -But this is still not optimal for machine learning.
%       -New non-von neumann<ref:?> architecture
%           -Look at GPGPU (general purpose computing on graphics processing unit)
%               -A "new" hardware which focuses more and data throughput<ref:>
%                and can gain a speedup of 10-100x times compare to running on a
%                CPU.(just check the examples out, don't link this: http://www.accelereyes.com/examples/case_studies)
%               -How many top500 computers are running with GPU's?
%                -And just to mention as a follow up that 2 major API's for
%                GPGPU has been developed since the release of the main article
%                (CUDA 08/openCL 09)<ref to official release documents>
%           -Non-perfect calculations <ref:?>
%           -Neural processors <ref:Synaptic> would be optimal at least for the
%           algorithms simulating neural networks (examples of algorithms)
%      
%       -New hardware => new algorithms
%           -Clusters
%           -For Map-reduce<ref>
%           -Hadoop with mahong (machine learning algorithms implemented (and
%           redesigned) to run effectively on clusters)
%
%   Artificial Scientist
%       -Autonomous experimentation
%       -Feedback to artificial intelligence itself (singularity)
%           -Genetic programming with a domain specific language like (example)
%       -Also improve the hardware it runs on
%       -I really hope we will be careful with the embodiment of these
%       artificial intelligence's so that we don't end up with Skynet<ref:
%       terminator> or a smart computer virus with a mind of their own<ref:?>.
%
%
%   Thoughts:
%       great refs and thoughts on the subject:
%       http://www.nature.com/nature/focus/futurecomputing/
%

\bibliographystyle{plain}
\bibliography{refs}

\end{document}
