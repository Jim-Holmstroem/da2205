\documentclass{article}
\usepackage{graphicx}
\usepackage{amsmath}
\usepackage{amssymb}
\usepackage{hyperref}
\usepackage{header}
\usepackage{natbib}
\usepackage[utf8]{inputenc}
\begin{document}

\generateheader{Final Assignment\\ \large ``Artificial Scientists and the
needed paradigm shift to get there''}

\section{Summary}
    This essay will be focusing on the current software and hardware limitations
    towards the development of new machine learning techniques a specially the
    artificial scientist. Much of the claims and discussions will be based on
    the article ``Towards 2020 Science'' from Microsoft Research.\cite{ms2020}

    It seems that we are currently standing even closer to the brink of a new 
    computer revolution. New machine learning techniques are starting to be 
    developed and set off into production. At least two big companies I have been working
    with, Ericcson and Oracle, are starting to consider using machine learning
    for tasks like for example abnormality detection for inhumanly complex
    systems with multiple soft
    states, classify the different states or different automatic utilities for
    testing software. This among other things results in
    the need of getting massive amounts of computing power.
    This hinder is nowadays much more focused on how to 
    distribute the computations to more computation units or sometimes use
    graphic cards but this is not a long time solution. We basically need a new
    computation paradigm for machine learning to get them as good as human
    performance in problems like object recognition or planing.

%    SUMMARY ON ARTIFICIAL SCIENTIST 


\section{The dawn of new hardware}
    The current processing capabilities for computers has lately degraded to a
    sub-exponential growth which is starting to have consequences such as 
    restricting the use of some algorithms as well as setting limits 
    to the processable size of datasets.
    This sub-exponential growth in computing has been a historical indicator
    that a new paradigm is on it's way which should be the case this time as
    well or at least there is a need for one.%<ref the_five_paradigm_changes_in_computing>
    
    The issue with most current hardware systems is their heavy and na\"{i}ve use of the von
    Neumann architecture. Basically the throughput for this types of processor
    has started to reach, for physical reasons, the end of the line 
    performance wise.

    The duck-tape solution to this has been keeping the von Neumann
    architecture, with it's limiting bandwidth, and just throw in more cores and/or more machines in parallel.
    This might work well in some big data applications where the problems is easily
    map-reducible but perform really badly in many other
    problems.\cite{mapreduce} Also
    this scaling becomes very space and power exhaustive since you have the need for
    cooling of all the cores. 

    The other type of solution is to drop the overused von Neumann architecture
    and go for something that is much more throughput oriented. In the 2020
    paper\cite{ms2020} they foresight
    \footnote{1 and 2 years before the official 
    announcement of CUDA resp. openCL.} 
    the use of graphical processing units
    (GPU) for general computing (GPGPU) that is problems with non graphical nature like
    solving differential equations or FFT. High end GPU's nowadays have around
    1500 stream processors in parallel which can perform one type of operation
    (kernel)
    on a 1500-element-chunk of the data in parallel at a time. This together
    with that use 
    of a read-only memory which removes the problem of locks
    and that all the buses are separated for
    maximum throughput they can in many cases  scale almost linearly 
    on the number of cores
    which ordinary CPU's almost never manages.

    Both scaling up with ordinary CPU's into multi- or manycore as well as
    GPGPU requires new libraries utilizing new programming paradigms.

    One possible solution instead of the old classical object oriented
    programming mostly used today would be to use a pure functional programming
    language such as
    for example Haskell. Some data driven companies has
    already conformed to this and are using it as their main language.
    \footnote{
        \href{http://www.janestreet.com}{Jane Street},
        \href{http://www.ericsson.com}{Ericsson},
        \href{http://www.campanja.com}{Campanja},
    }
    The inherited properties that makes functional programming languages 
    preferable is for example the lack of state 
    which makes it lack execution order which in turn makes it
    trivial to convert it to run it in parallel, since it already run in parallel
    semantically. Both also the extremely high level of the language with
    compositions and higher order functions built-in.\cite{haskell}

    But I still think that an entirely new to be developed even if the GPU
    solves many of the problems.
    
    Probably the easiest way trying to imitate human behavior, which in many
    cases outperforms computers, is to actually
    imitate human processing. A new paradigm that has recently emerged is neuromorphic
    computing which basically tries to mimic neural functions with electronics.
    
    The freshest and most promising according to me is the 
    neural processing unit (NPU) which both IBM\cite{synapse} and
    Intel\cite{intelneuro} are starting develop. Both the approaches avoids being
    just a monstrous special purpose VHDL-logic chip and can thus be thought of
    as a new paradigm over neuromorphic computing.

    Just as for GPGPU's the usage of a NPU's requires a new programming paradigm to
    operate, not much information on the NPU's nor their development tools
    have been released but I can speculate that it would require some variant of
    logical- or functional programming. Also the algorithms that fits into the
    NPU's needs to be greatly rewritten.
    
    Another rather new and interesting neumorphic device is the
    dynamic vision sensor (DVS) which basically has asynchronous event based
    readout by spiking on color changes.\cite{dvs} It works much more like the human retina and
    these properties makes it easier to do 3D reconstruction as well as easily
    being really fast without using large bandwidth nor much power.

    In addition to all this some more radical improvements in learning
    algorithms has to be made as well. A very recent shift towards deep
    learning algorithms in the machine learning community that probably can be
    described as a paradigm shift. 
   
    Much of the recent success in deep learning is due to the new found computing
    power from 
    GPU's as I mentioned before as well as a new look on deep learning. A great
    example the later is dropout algorithm on deep networks which not only
    outperforms the rest on the podium but also is much less ad hoc by
    basically only log some of the features.\cite{dropout}

\section{A new breed of scientists}
    The need of new hardware becomes really apperant both in
    theory and practice when trying to create a general intelligence.
    A few really recent and in many ways successful tries has been made
    like for example: Google's unsupervised catdetector %REF catdetector
    %IBM simulation Synapse (distinguish this with the NPU Synapse)
    These tries has been run on large clusters blabla but would gain a
    significant effeciency boost being run on more natural habitat namely the
    NPU. 


    \bibliographystyle{plain}
    \bibliography{refs}

\end{document}
