\documentclass{article}
\usepackage{graphicx}
\usepackage{amsmath}
\usepackage{amssymb}
\usepackage{hyperref}
\usepackage{header}
\begin{document}

\generateheader{Response 1}
    This is a response to the chapter `The Scientific Process' from
    'Advice to a Young Scientist' by Peter Medawar. It can be summarized as going
    trough and discussing the aspects of a scientific process at different
    levels, such as; testing hypotheses by experimentation, using logic 
    systems for deduction, and
    the self-correcting process of critical evaluation.

    It seems by the text that machine learning in many ways and at many levels tries 
    to mimic 
    the scientific process by, for example: setting up and testing hypotheses by
    experimentation (collecting data) and then refine them 
    according to the data (which can lead to overfitting the model, see below), 
    and repeating this process until it reaches a good enough model of the 
    data. One problem with this search for a solution is the very
    vastness of the most general hypothesis space. Where humans uses the 
    "hunch" to guide the hypothesis search the algorithms fall short by lacking 
    the possibility of using this concept and instead needs to use a constructed
    subspace.

    The chapter also touches on the subject of trying to fit a hypothesis 
    to seen data and not the other way around. Doing it the other way around 
    can be problematic especially when your working with sparse data since it 
    is then more likely for the hypothesis under test to be perceived as good 
    solely by the effect of coincidence, that is your hypothesis has fitted the
    noise from the data.

\end{document}
