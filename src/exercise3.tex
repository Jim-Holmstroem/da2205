\documentclass{article}
\usepackage{graphicx}
\usepackage{amsmath}
\usepackage{amssymb}
\usepackage{hyperref}
\usepackage{header}
\begin{document}

\generateheader{Response 3}
   The chapter on Turing describes the process in which the Turing machine was
   developed and why as well as a quite extensive description on how the Turing machine
   operates.

   The entire thing started out as a tool in the hunt to show that
   entschielöngsproblems can't generally be solved by a machine. but turned out that the
   concepts used was ideal to be used in computers. Handing a machine a
   "program" (was this a Turing discovery?) what has computability to do with
   this? 

\end{document}
