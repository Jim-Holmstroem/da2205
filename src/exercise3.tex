\documentclass{article}
\usepackage{graphicx}
\usepackage{amsmath}
\usepackage{amssymb}
\usepackage{hyperref}
\usepackage{header}
\begin{document}

\generateheader{Response 1}

This is a response to the chapter on `The Scientific Process' in 
Advice to a Young Scientist by Peter Medawar. It can be summarized as going
trough and discussing the aspects of a scientific process, at different levels,
such as; testing
hypothesises by experimentation, using logic systems for deduction (with the
possibility of asymmetry), as well as the self-correcting process of critical 
evaluation.

It seems by the chapter that machine learning in many ways tries to mimic the scientific
process by, for example, setting up and testing hypothesis by
experimentation and then refine these hypothesis according to the data and repeating
this process
until it reaches a good enough model of the data. One problem with this search
for truth is the very
vastness of the most general hypothesis space, where humans uses the "hunch" to
guide the search but algorithms often fall short by lacking the possibility
to use this concept, at least in the more general sense.

The chapter also touches on the subject with trying to fit a hypothesis to seen
data and not the other way around. This can be faulty especially when your
data is sparse making it more plausible for a given hypothesis to
be perceived as true solely by the effect of coincidence.

\end{document}
